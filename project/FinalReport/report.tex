\documentclass[times]{article}

\usepackage[margin=1.0in]{geometry}
\usepackage{graphicx}
\usepackage{adjustbox}
\usepackage{float}
\usepackage{placeins}
\usepackage[none]{hyphenat}
\usepackage{amsmath}
\usepackage[us]{datetime}
\usepackage[explicit]{titlesec}
\usepackage{url}
\usepackage{siunitx}
\usepackage{multirow}
\usepackage{rotating}

% Dataset: http://archive.ics.uci.edu/ml/datasets/Heart+Disease

\begin{document}
   \title{Stat 5353  - Fall 2017 \\ Final Report}
   \author{Dalton Cole \\ Adam Harter \\ Samuel Richter}
   \date{\formatdate{3}{12}{2017}}
   \maketitle
   
   \section{Introduction}

   In this experiment, k-means clustering was applied to the UCI heart disease data 
   set\footnote{http://archive.ics.uci.edu/ml/datasets/Heart+Disease}, with the factors 
   being the distance formula used and the number of clusters formed.  
   The three levels for the distance formula used were Euclidean distance, 
   Cosine distance, and Jaccard distance.  
   The three levels for the number of clusters were 2, 3, and 5.  
   The response variable of interest is the resulting sum squared error of the clusters.

   \section{Design}

   The design of the experiment was a completely randomized design.
   To achieve this, every possible combination of the factors was run in random order,
   with each combination being run twice.
   The initial starting points for each cluster was also initialized randomly.
   ``Environmental Error'' is introduced by randomly choosing data points to train on and
   data points to test against.

   \section{Procedure and Data Collection}

   To collect data, k-means clustering was performed using the Python library
   nltk\footnote{http://www.nltk.org/\_modules/nltk/cluster/kmeans.html}.
   The data for the k-means clustering itself was taken from the UCI heart disease
   data set and each point contained 14 used attributes, which are as follows:

   \begin{enumerate}
      % reduce spacing between items
      \itemsep0em
      \item Age
      \item Sex
      \item Chest Pain Type
      \item Resting Blood Pressure
      \item Cholesterol Level
      \item Fasting Blood Sugar $>$ 120 mg/dl
      \item Resting ECG Results (normal, ST-T wave abnormality, left ventricular hypertrophy)
      \item Maximum heart rate achieved
      \item Exercise induced Angina
      \item ST Depression Induced by Exercise Relative to Rest
      \item Slope of Peak Exercise ST Segment (up-sloping, flat, down-sloping)
      \item Number of Major Vessels Colored by Flourosopy
      \item Thalassemia
      \item Diagnosis of Heart Disease
   \end{enumerate}

   The full list of data used can be found in Table~\ref{tab:raw_data}

   \section{Analysis of Results}

   \begin{table}[H]
      \centering
      \caption{ANOVA Table}
      \label{tab:anova}
      \begin{tabular}{| c | c | c | c | c |}
         \hline
         Source              & d.f.   & SS        & MS      & F-ratio    \\
         \hline
         Model               & 8      & 16247.444 & 2030.93 & 44.4189    \\
         \hline
         Error               & 9      & 411.500   & 45.72   & Prob $>$ F \\
         \hline
         Combination Total   & 17     & 16658.944 &         & $<$.0001   \\
         \hline
      \end{tabular}
   \end{table}

   \begin{table}[H]
      \centering
      \caption{Effect Tests}
      \label{tab:effect}
      \begin{tabular}{| c | c | c | c | c | c |}
         \hline
         Source & 
            Nparm & DF &        SS & F Ratio  & Prob $>$ F \\
         \hline
         Number of Clusters &
            2     & 2  & 15786.111 & 172.6306 & $<$.0001 \\
         \hline
         Distance Metric &
            2     & 2  &   100.000 &   1.0936 & 0.3757 \\
         \hline
         Distance Metric * Number of Clusters &
            4     & 4  &   374.222 &   2.0462 & 0.1711 \\
         \hline
      \end{tabular}
   \end{table}
   
   \begin{table}[H]
      \centering
      \caption{Tukey's test ($\alpha = 0.05$)}
      \label{tab:tukey}
      \newcommand{\singlecol}[1]{\multicolumn{1}{c|}{#1}}
      \newcommand{\doline}{\cline{2-5}}
      \begin{tabular}{ c | l | S | S | S |}
         \multicolumn{1}{c}{} & \multicolumn{4}{c}{LSMean[j]} \\
         \doline
         \parbox[t]{2mm}{\multirow{16}{*}{\rotatebox[origin=c]{90}{LSMean[i]}}}
         & Mean[i] - Mean[j] & \singlecol{2} & \singlecol{3} & \singlecol{5} \\
         & Std Err Dif & & & \\
         & Lower CL Dif & & & \\
         & Upper CL Dif & & & \\
         \doline
         & 2 & 0 & 38.3333 & 72.5 \\
         &   & 0 & 3.90394 & 3.90394 \\
         &   & 0 & 27.4335 & 61.6002 \\
         &   & 0 & 49.2332 & 83.3998 \\
         \doline
         & 3 & -38.333 & 0 & 34.1667 \\
         &   & 3.90394 & 0 & 3.90394 \\
         &   & -49.244 & 0 & 23.2668 \\
         &   & -27.433 & 0 & 45.0665 \\
         \doline
         & 5 & -72.5 & -34.167 & 0 \\
         &   & 3.90394 & 3.90394 & 0 \\
         &   & -83.4 & -45.067 & 0 \\
         &   & -61.6 & -23.267 & 0 \\
         \doline
      \end{tabular}
      
      \begin{tabular}{| c | c | c | c | c |}
         \hline
         Level &   &   &   & Least Sq Mean \\
         \hline
         2     & A &   &   & 180.00000 \\
         \hline
         3     &   & B &   & 141.66667 \\
         \hline
         5     &   &   & C & 107.50000 \\
         \hline
      \end{tabular}
   \end{table}

   \section{Conclusion}

   The ANOVA table can be found in Table~\ref{tab:anova}.
   Using $\alpha = 0.05$, the different combinations of factors was significant, as
   Prob $>$ F $<$ .0001, which is less than 0.05.
   Using the effect tests table, found in Table~\ref{tab:effect}, several things can be concluded.
   There is no interaction between the Distance Metric and the Number
   of Clusters, as Prob $>$ F = 0.1711, which is greater than 0.05.
   The Distance Metric was also not significant as Prob $>$ F = .3757, which is greater than 0.05.
   The number of clusters, however, was significant, as Prob $>$ F $<$ .0001, which is less than 0.05.
   Each number of clusters was statistically distinct and having two clusters produced the most 
   accurate predictor, as shown in Table~\ref{tab:by_cluster} and Table~\ref{tab:tukey}.  
   Tukey's test was performed on the number of clusters instead of a regression as the number of
   clusters is discrete, and not continuous.
   
   \begin{table}[H]
      \centering
      \caption{Number Correct by Cluster Size}
      \label{tab:by_cluster}
      \begin{tabular}{| c | c |}
         \hline
         Number of Clusters & Number Correct \\
         \hline
         2 & 1080 \\
         \hline
         3 & 849 \\
         \hline
         5 & 645 \\
         \hline
      \end{tabular}
   \end{table}

   \appendix
   \section{List of Data}

   \begin{table}[H]
      \centering
      \caption{Experimental Data}
      \label{tab:data}
      \begin{tabular}{| c | c | c |}
         \hline
         Distance Metric   & Number of Clusters   & Number Correct \\
         \hline
         Euclidean         & 2                    & 180            \\
         \hline
         Euclidean         & 3                    & 139            \\
         \hline
         Euclidean         & 5                    & 105            \\
         \hline
         Cosine            & 2                    & 184            \\
         \hline
         Cosine            & 3                    & 149            \\
         \hline
         Cosine            & 5                    & 104            \\
         \hline
         Jaccard           & 2                    & 176            \\
         \hline
         Jaccard           & 3                    & 151            \\
         \hline
         Jaccard           & 5                    & 108            \\
         \hline
         Euclidean         & 2                    & 180            \\
         \hline
         Euclidean         & 3                    & 138            \\
         \hline
         Euclidean         & 5                    & 99             \\
         \hline 
         Cosine            & 2                    & 186            \\
         \hline
         Cosine            & 3                    & 137            \\
         \hline
         Cosine            & 5                    & 101            \\
         \hline
         Jaccard           & 2                    & 174            \\
         \hline
         Jaccard           & 3                    & 136            \\
         \hline
         Jaccard           & 5                    & 128            \\
         \hline
      \end{tabular}
   \end{table}

   \begin{table}[H]
      \centering
      \caption{Heart Diesease Data}
      \label{tab:raw_data}
      \begin{tabular}{rrrrrrrrrrrrrr}
         \hline
         Age & Sex & Chest Pain & Blood Pressure & Cholesterol & Blood Sugar & 
            ECG & Heart rate & Angina & ST Depression & ST Slope & Major Vessels 
            & Thalassemia & Diagnosis \\
         \hline
         \input{raw.txt}
      \end{tabular}

\end{document}
