\documentclass[times]{article}

\usepackage[margin=1.0in]{geometry}
\usepackage{graphicx}
\usepackage{adjustbox}
\usepackage{float}
\usepackage{placeins}
\usepackage[none]{hyphenat}
\usepackage{amsmath}
\usepackage[us]{datetime}
\usepackage[explicit]{titlesec}
\usepackage{url}
\begin{document}
	\title{Stat 5353  - Fall 2017 \\ Project Proposal}
	\author{Dalton Cole \\ Adam Harter \\ Samuel Richter}
	\date{\formatdate{13}{10}{2017}}
	\maketitle
	
	For this project, we propose applying k-means clustering to the UCI heart disease data set \cite{ref:uci}. 
   Zero represents no heart disease, and 1-5 represents different levels of heart disease.
   Our factors will be: distance formula used and number of clusters formed. 
   The three levels of distance formulas will be: euclidean distance, manhattan distance, and Chebyshev distance. 
   The three levels of the number of clusters will be 2, 3, and 5. 
   The purpose of the number of clusters is that the data set has 5 different levels of heart disease.
   Due to the low number of total combinations, all treatment combinations will be tested in random order.
   Each combination will be run twice, with the second run occurring directly after the first.
   The initial starting points for each cluster will be determined randomly, thus introducing ``environmental'' error. 
   The response variable of this experiment is the resulting sum squared error of the clusters.

	We will be using the nltk library in python to perform k-means \cite{ref:nltk}. 
   After creating the model, we will test the model against a set aside test set. 
   The sum of squared error or the actual value and the predicted value will be the final result. 
   Table \ref{tab:anova} shows the corresponding ANOVA table.

	\begin{table}[!h]
		\centering
		\caption{ANOVA Table}
		\label{tab:anova}
		\begin{tabular}{| c | c | c | c | c |}
			\hline
			Source 					& d.f.	& SS	& MS 	& F-ratio \\
			\hline
			Treatment Combinations	& 8		& $SS_{Treat Comb}$	& $MS_{Treat Comb}$	& $F_{Trt}$		\\
			\hline
			Number of Clusters (NC)	& 2		& $SS_{NC}$			& $MS_{NC}$			& $F_{NC}$ 		\\
			\hline
			Distance Formula (DF)	& 2		& $SS_{DF}$			& $MS_{DF}$			& $F_{DF}$ 		\\
			\hline
			NC * DF					& 4 	& $SS_{NC DF}$		& $MS_{NC DF}$		& $F_{NC DF}$ 	\\
			\hline
			Error					& 9 	& SSE				& MSE				& 				\\
			\hline
			Total					& 17 	& SSTotal			& 					& 				\\
			\hline
		\end{tabular}
	\end{table}

	\medskip
	\bibliographystyle{plain}
	\bibliography{bib}

\end{document}